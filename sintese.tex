%%%%%%%%%%%%%%%%%%%%%%% file typeinst.tex %%%%%%%%%%%%%%%%%%%%%%%%%
%
% This is the LaTeX source for the instructions to authors using
% the LaTeX document class 'llncs.cls' for contributions to
% the Lecture Notes in Computer Sciences series.
% http://www.springer.com/lncs       Springer Heidelberg 2006/05/04
%
% It may be used as a template for your own input - copy it
% to a new file with a new name and use it as the basis
% for your article.
%
% NB: the document class 'llncs' has its own and detailed documentation, see
% ftp://ftp.springer.de/data/pubftp/pub/tex/latex/llncs/latex2e/llncsdoc.pdf
%
%%%%%%%%%%%%%%%%%%%%%%%%%%%%%%%%%%%%%%%%%%%%%%%%%%%%%%%%%%%%%%%%%%%


\documentclass[runningheads,a4paper]{llncs}

% inserido por mim
\usepackage[utf8]{inputenc}
\usepackage[T1]{fontenc}
\usepackage{lmodern} % load a font with all the characters
\usepackage[portuguese]{babel}

\usepackage{amssymb}
\setcounter{tocdepth}{3}
\usepackage{graphicx}

\usepackage{url}
\urldef{\mailsa}\path|j.campinhos@campus.fct.unl.pt,|
\urldef{\mailsb}\path|p.duraes@campus.fct.unl.pt|
\newcommand{\keywords}[1]{\par\addvspace\baselineskip\noindent\keywordname\enspace\ignorespaces#1}

\begin{document}

\mainmatter% start of an individual contribution

% first the title is needed
\title{Novas fronteiras da ética na Informática:\\Inteligência Artificial e Agentes Autónomos}

% a short form should be given in case it is too long for the running head
\titlerunning{Inteligência Artificial e Agentes Autónomos}

% the name(s) of the author(s) follow(s) next
%
% NB: Chinese authors should write their first names(s) in front of
% their surnames. This ensures that the names appear correctly in
% the running heads and the author index.
%
\author{João Campinhos e Pedro Durães}
%
\authorrunning{Inteligência Artificial e Agentes Autónomos}
% (feature abused for this document to repeat the title also on left hand pages)

% the affiliations are given next; don't give your e-mail address
% unless you accept that it will be published
\institute{Faculdade de Ciências e Tecnologia da Universidade Nova de Lisboa,\\
Quinta da Torre, 2829 -- 516 Caparica, Portugal\\
\mailsa\\
\mailsb\\
\url{http://http://www.fct.unl.pt}}

%
% NB: a more complex sample for affiliations and the mapping to the
% corresponding authors can be found in the file "llncs.dem"
% (search for the string "\mainmatter" where a contribution starts).
% "llncs.dem" accompanies the document class "llncs.cls".
%

\toctitle{Inteligência Artificial e Agentes Autónomos}
\tocauthor{Inteligência Artificial e Agentes Autónomos}
\maketitle


\begin{abstract}
Este documento visa analisar o tema da inteligência artificial e agentes autónomos no âmbito dos aspectos sócio-profissionais da informática.
Iremos abordar não só a forma como a inteligência artificial pode ajudar a humanidade, mas também os problemas éticos e riscos que levanta.
Com o desenvolvimento de agentes autónomos como nos carros sem condutor, deparamo-nos com inúmeros riscos que precisam de ser ponderados, para que o desenvolvimento da inteligência artificial não seja prejudicial.
\keywords{As palavras chave são aqui!}
\end{abstract}

\section{Introdução}

A inteligência artificial está muito associada a robôs pois este é um tema bastante abordado no cinema. Geralmente nesses filmes, os engenheiros que programam estes agentes autónomos criam algo maior que eles próprios e que deixam de poder controlar. Mas isto é ficção, e o nosso objectivo com este documento é tentar aproximar este tema da realidade, pois apesar de ainda estarmos numa fase bastante embrionária no desenvolvimento de agentes autónomos, é uma área potencialmente perigosa, e cabe-nos a nós, engenheiros e futuros engenheiros informáticos, com um desenvolvimento responsável fazer com que o mau da inteligência artificial nunca venha ao de cima.

Existe ainda uma certa relutância e negação quando é abordado este tema, mas na verdade cada vez mais pessoas importantes na área estão a levá-lo a sério, como é o caso do Elon Musk, cofundador do Paypal e Tesla Motors e fundador da empresa de exploração espacial SpaceX, que doou 10 milhões de dólares para o instituto Future of Life Institute, numa tentativa de promover um desenvolvimento da inteligência artificial e agentes autónomos estritamente beneficiais para a humanidade.

\section{Contexto}

Em 1942, foram introduzidas numa obra de ficção de Isaac Asimov as famosas três leis da robótica:

\begin{enumerate}
    \item Um robô não pode ferir um ser humano ou, por inacção, permitir que um ser humano sofra algum mal.
    \item Um robô deve obedecer as ordens que lhe sejam dadas por seres humanos excepto nos casos em que tais ordens entrem em conflito com a Primeira Lei.
    \item Um robô deve proteger sua própria existência desde que tal protecção não entre em conflito com a Primeira ou Segunda Leis.
\end{enumerate}

Estas leis começam por levantar um problema importantíssimo na inteligência artificial: o mal que um robô pode fazer ao ser humano. É claro que, em 1942 esse não era um problema, mas na actualidade, e com o desenvolvimento da inteligência artificial, começamos a ter de pensar nestes casos, e analisar o problema de forma objectiva.

À primeira vista, as três leis de Isaac Asimov parecem ser bastante estritas e claras, e ter um robô a respeitá-las parece ser bastante seguro. O que é certo é uma analise mais cuidada levanta algumas ambiguidades importantes. A começar pela própria definição de robô e ser humano, que pode ser mal interpretada pelo robô fazendo com que não respeite as três leis. Outro problema reside no facto do robô poder quebrar uma lei sem se aperceber. O que acontece nessa situação? E existem mais ambiguidades, mas só por aqui podemos ver a complexidade da inteligência artificial, e a importância.

\begin{thebibliography}{4}

\end{thebibliography}

\end{document}
